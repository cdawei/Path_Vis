% !TEX root = ./acmmm2017.tex

\section{Introduction}
%background
Sequence recommendation has emerged as an important framework for modelling diverse problems such as travel route and music playlist recommendation~\cite{chen2017SR}. 
Unlike classical ranking algorithms where items are considered independently~\cite{koren2009matrix},
a sequence recommendation algorithm requires modelling a structure between items and suggests a set of items as a whole. 
For example, consider recommending a trajectory of \emph{points-of-interest} (POIs) in a city to a visitor. 
While a classical ranking algorithm can learn a user's preference for each individual location, it may ignore the distances between them and could suggest a longer trajectory than is optimal. 
Several sequence recommendation algorithms have been proposed to solve this problem and demonstrated superior performance compared to classical ranking algorithms~\cite{ijcai15,chen2017SR}. 
Nonetheless, recommendation algorithms for sequences and trajectories~\cite{chen2016learning,chen2017SR} have many components and can be difficult for a user to understand. This is part of the general challenge of introducing transparency and accountability for machine learning algorithms~\cite{fatml}. 
%a remaining challenge is to construct an interactive recommendation system so that a user can analyse the suggested sequences and plan a better trip.

%approach
In this paper, we tackle the problem of sequence visualisation, specifically focussing on travel routes recommendation. 
%We define a travel route as a sequence of POIs and formulate the sequence recommendation problem as a structured prediction problem. 
A travel route is a sequence of POIs, and the sequence recommendation problem can be formulated as a structured prediction problem~\cite{chen2017SR}.
Based on a diverse set of features for individual and pairs of POIs, we train the prediction model with trajectory data extracted from geo-tagged photos taken in Melbourne~\cite{chen2016learning}. 
To visualise the suggested routes, we develop a novel tool that efficiently displays multiple suggested routes, which helps users understand the process behind the recommendations.
Specifically, our system decomposes a total score of each route into a set of features and their corresponding scores, and shows the total score as a stacked bar plot of the features.
The system also visualises the differences between POIs in a single route to show how POIs in that route can exhibit vast diversity. 

This visualisation helps tourists who want %to have
diverse experiences by choosing the best route among multiple recommendations. %the set of recommendations. 
Generalising to a broader class of routes, such a visualisation could also help users of online mapping apps to make decisions on suggested travel routes, such as by trading off distance, traffic, and scenery. 
