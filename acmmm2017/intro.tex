% !TEX root = ./acmmm2017.tex

\section{Introduction}
%background
Sequence ranking has emerged as an important tool for solving diverse problems such as travel route and music playlist recommendations. 
Unlike the classical ranking algorithm where items are considered independently, the sequence ranking algorithm requires modelling a structure between items and suggests a set of items as a whole. 
For example, consider recommending a trajectory of points of interest (POI) in a city to a visitor. 
While the classical ranking algorithm can learn a user's preference for each individual location, it may ignore the distances between them and could suggest a long trajectory, which should be shorter in optimal routing. 
Several sequence ranking algorithms have been proposed to solve this problem and demonstrated superior performance compared to classical ranking algorithms. 
Nonetheless, recommendation algorithms for sequences and trajectories~\cite{chen2016learning,chen2017SR} have many components and can be difficult for a user to understand. This is part of the general challenge of introducing transparency and accountability for machine learning algorithms~\cite{fatml}. 
%a remaining challenge is to construct an interactive recommendation system so that a user can analyse the suggested sequences and plan a better trip.

%approach
In this paper, we tackle the problem of sequence visualisation, specifically focussing on travel routes recommendation. 
We define a travel route as a sequence of POIs and formulate the sequence ranking problem as a structured prediction problem. 
Based on a diverse set of features for individual and pairs of POIs, we train the prediction model with trajectory data extracted from geo-tagged photos. 
To visualise the suggested routes, we develop a novel tool that efficiently displays multiple suggested routes, which helps users understand the process behind the recommendations.
Specifically, our system decomposes a total score of each route into a set of features and their corresponding scores and shows the total score as a stacked bar plot of the features.
The system also visualises the difference between POIs in a single route to show how POIs in that route can exhibit vast diversity. 
This visualisation helps tourists who want to have diverse experiences by choosing the best route among the set of recommendations. Generalising to the broader class of routes, such a visualisation could also help users of online mapping apps to make decisions on recommended travel routes, such as trading off distance, traffic, and scenery. 
