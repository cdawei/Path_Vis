\documentclass[sigconf]{acmart}
\settopmatter{printacmref=false}
\setcopyright{none}

\begin{document}

\title{PathRec: Visual Analysis of Travel Route Recommendations}
\subtitle{Supplementary Material}
\maketitle

\thispagestyle{empty}

\appendix

\section{Alternative approaches to trajectory recommendation}
\label{sec:alternative}

A number of approaches have been proposed to solve the trajectory recommendation problem.
\citet{ijcai15} formulated an optimisation problem inspired by the travelling salesman problem,
and \citet{cikm16paper} proposed to learn a RankSVM~\cite{lranksvm} model to rank POIs with respect to a query,
in particular, 
the training objective is
\begin{equation*}
\resizebox{\linewidth}{!}{$\displaystyle
\min_{\mathbf{w}} \frac{1}{2} \mathbf{w}^\top \mathbf{w} + C \cdot \sum_{i=1}^n \sum_{(p,p') \in \mathcal{R}(\mathbf{x}^{(i)})}
          \max\left(0, 1 - \mathbf{w}^\top \left(\Phi(\mathbf{x}^{(i)}, p) - \Phi(\mathbf{x}^{(i)}, p') \right) \right)^2,
$}
\end{equation*}
where $\mathbf{w}$ denotes the model parameters, $\Phi$ is a query-POI feature mapping, $C > 0$ is a regularisation constant,
and $\mathcal{R}(\mathbf{x})$ is the set of POI pairs $(p, p')$ such that $p$ is ranked above $p'$,
e.g. POI $p$ is observed more often than POI $p'$, with respect to query $\mathbf{x}$.
Lastly, the top-ranked POIs with respect to the given query were taken to form a trajectory.


\bibliographystyle{ACM-Reference-Format}
\bibliography{ref_appendix}

\end{document}
