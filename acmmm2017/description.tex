\documentclass[sigconf]{acmart}
\settopmatter{printacmref=false}
\setcopyright{none}

\begin{document}

\title{Description of the demonstrated system}
\maketitle

\thispagestyle{empty}

%\section{Description of the demonstrated system}
%\label{sec:desc-sup}
% description of the demo system
\section{System overview}
\label{sec:overview}
The demonstrated system comprises two components, i.e. the server side and client side.
At the server side, a set of travel routes are generated from a trained structured SVMs with respect to
a query sent by the client. 
The generated routes are then sent to the client, which runs in a web browser, to visualise.
The server side was implemented in the Python programming language using \textit{pystruct}~\cite{JMLR:v15:mueller14a},
a Python library for structured learning and prediction,
with the help of other essential Python libraries (see Section~\ref{sec:setup} for details).
The client side was implemented in HTML, JavaScript and CSS with the help of \textit{GMaps.js}~\cite{gmaps.js}, \textit{LineUp.js}~\cite{lineup.js} and a radar chart JavaScript library~\cite{radarchart.js}.


% required setup
% website for users
\section{System setup}
\label{sec:setup}
We created a website for users to try this system, it can be accessed at \url{http://115.146.87.43:8080/}.
The source code of this system is hosted in a public GitHub repository, 
so anyone interested in it can also run the system locally, by following the steps below:
\begin{enumerate}
\item Download the source code of the system from \url{https://github.com/cdawei/path_vis} 
% DW: move this to https://github.com/computationalmedia if it gets accepted?
\item Install the Python runtime and required libraries: 
      \begin{itemize}
      \item \textit{Python} version 3.5.3
      \item \textit{numpy}~\cite{numpy} version 1.12.1 or newer
      \item \textit{scipy}~\cite{scipy} version 0.19.0 or newer
      \item \textit{pandas}~\cite{pandas} version 0.19.2
      \item \textit{scikit-learn}~\cite{sklearn} version 0.18.1 or newer
      \item \textit{joblib}~\cite{joblib} version 0.11 or newer
      \item \textit{pystruct}~\cite{JMLR:v15:mueller14a} version 0.2.4 or newer
      \end{itemize}
\item Go to the directory with system source code and launch the server side by typing \texttt{python server.py}
\item Access \url{http://localhost:8080} using a web browser to use the visualisation system.
\end{enumerate}


%information about the presenters, including the relationship to the project
\section{Presenter}
\label{sec:presenter}
The demonstrated system will be presented by Dawei Chen, a second year PhD student at the Australian National University.
The presenter is one member in a group that working on this project.


\bibliographystyle{ACM-Reference-Format}
\bibliography{desc} 

\end{document}
